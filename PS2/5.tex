\begin{solution}{Question 5}\label{ques:5}
    \begin{question}
    A 2-NFA $A$ is a 5-tuples $A=(Q,S,t,\Sigma,\Delta)$ where $Q$ is the set of states, $S$ the set of start states, $t$ is an accept state, the transition function
    $$\Delta: Q \times (\Sigma \cup \{\#, \$\}) \to 2^{Q \times (\{L,R\}) } $$
    Assume that whenever $M$ accepts, it does so by moving the head (pointer) all the way to the right end marker \$ and entering accept state t. In the subsequent two questions, we will try to prove 2-NFAs accept only regular languages.
    
    \begin{enumerate}
        \item Let $x = a_1\dots a_n \in \Sigma^*, a_i \in \Sigma, 1 \leq i \leq n.$ Let $a_0 = \#, a_{n+1} = \$.$ Argue that $x$ is not accepted by $A$ if and only if there exists sets $W_i \subseteq Q, 0 \leq i \leq n+1$ such that the following hold:
        \begin{itemize}
            \item $S \subseteq W_0$ \label{1}
            \item If $u \in W_i, 0 \leq i \leq n \text{, and } (v,R) \in \Delta (u,a_i), \text{ then } v \in W_{i+1}$
            \item If $u \in W_i, 1 \leq i \leq n+1 \text{, and } (v,L) \in \Delta (u,a_i), \text{ then } v \in W_{i-1}$
            \item $t \not\in W_{n+1}$
        \end{itemize}
        \item Using the previous part, show that $L(A)$ is regular
    \end{enumerate}
    \end{question}
    \tcblower{}
    \begin{proof}[Solution: Part 1A]
    $(\implies)$ We assume that $x$ is not accepted\\
    For each run of $A$, we keep a track of the states that we end up in before reading each symbol in $x$. Let us denote the set of states reached before reaching symbol $a_i$ for run $j$ as $(S_j)_i$. There are a finite number of runs for each $x$ and therefore $j$ is bounded by say $k$. We now define the sets $W_i$ as:
    \begin{equation}
        W_i = \bigcup_{j=1}^k (S_j)_i
    \end{equation}
    Now, we prove all the four conditions:
    \begin{enumerate}
        \item $S \subseteq W_0$: This is trivially true since at least one run exists where $A$ begins in each start state.
        \item If $u \in W_i, 0 \leq i \leq n \text{, and } (v,R) \in \Delta (u,a_i), \text{ then } v \in W_{i+1}$: If for a run $S_j$ of $x$, we get the transition $(v, R)$, we then read $a_{i+1}$. Therefore, $v$ will be in the set $(S_j)_{i+1}$ and thus in the state $W_{i+1}$
        \item If $u \in W_i, 1 \leq i \leq n+1 \text{, and } (v,L) \in \Delta (u,a_i), \text{ then } v \in W_{i-1}$: Similar to the previous part, this will be true.
        \item $t \not\in W_{n+1}$: Since $x$ is not accepted by $A$, therefore, no run of $A$ on reading $x$ will end on $t$ on reaching $a_{n+1} = \$$ (just before reading $\$$). Therefore, $t \not\in W_{n+1}$
    \end{enumerate}
    \end{proof}
    
    \begin{proof}[Solution: Part 1B]
    ($\Longleftarrow$) all conditions are true, show x is rejected.\\
    
    First condition says that $S$ is an improper subset of $W_0$. Second and third condition set up the sets $W_is$ using R and L on given states. Since the number of runs for $x$ is finite, let's say that $x$ has $k$ runs. Let the set of states reached just before reaching symbol $a_i$ in run $j$ be $(S_j)_i$.
    
    Using second and third, we observe that $W_i$ is the set of all such states we can be present in before reading the $i$-th symbol of $x$, i.e., $a_i$. After reading the $i$-th symbol, we can go to a state which is in either of $W_{i-1}$ or $W_{i+1}$. Therefore, $W_i$ is exactly equal to:
    \begin{equation}
        W_i = \bigcup_{j=1}^k (S_j)_i
    \end{equation}
    Following this formula, $W_{n+1}$ contains all the possible states where $x$ can end up in on arriving at $a_{n+1} = \$$ in its runs. Since $t \not\in W_{n+1}$, $x$ has never ended up in state $t$ (when at $\$$) which is the only accepting state. Hence $x$ is rejected.
    \end{proof}
    
    \begin{proof}[Solution: Part 2]
        We construct an NFA $N = (Q', \Sigma, \delta, q_0, F)$ for $\overline{L(A)}$ as follows:
        \begin{equation}
            \begin{split}
                Q' &= 2^Q \cup \{q_0\}\\
                \delta(S, a) &= \{T\ |\ \exists\ x \in \Sigma^* : \exists\ 0\leq i \leq |x|+1: S = W_i(x) \wedge T = W_{i+1}(x)\}\\
                \delta(q_0, \epsilon) &= \{S\ |\ \exists\ x \in \Sigma^* : S = W_0(x)\}\\
                F &= \{S\ |\ \exists\ x \in \Sigma^* : t \not\in S \wedge S = W_{|x|+1}(x)\}
            \end{split}
        \end{equation}
        It is easy to see that the above definition defined an NFA. In words, it effectively defines outward transitions for the right closure and incoming transitions for the left closure of the sets $W_i$ defined in the previous part. We now prove the correctness formally.\par
        On reading any word $x$, if it is not accepted by $A$, i.e., $x \not\in L(A) \implies x \in \overline{L(A)}$, we can generate a sequence of states $W_0, W_1, \ldots, W_{|x|+1}$ such that $t \not\in W_{|x|+1}$. From the definition of $N$, it is also easy to see that a run of $N$ exists where we end up at state $W_{|x|+1}$ and this state $W_{|x|+1}$ will be in $F$ (since $t \not\in W_{|x|+1}$). Therefore, $x \in L(N)$. Also it is easy to see that any word $x$ accepted by $N$ will be in $\overline{L(A)}$.\par
        Therefore, we have constructed an NFA $N$ for $\overline{L(A)}$. Therefore, $\overline{L(A)}$. From the closure of regular languages under complement, $L(A)$ is also regular. Hence, proved.
    \end{proof}
\end{solution}
