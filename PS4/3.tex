\begin{solution}{Question 3}\label{ques:3}
    \begin{question}
        Let $C$ be a language. Prove that C is Turing-recognizable iff a decidable language $D$ exists such that
        $$C = \{ x | \exists y (\langle x,y \rangle \in D) \}$$
    \end{question}
    \tcblower{}
    \begin{proof}[Solution]
    %%%%%%%%%%%%%%%%%%%%%%%%%%%%%%%
    To prove the iff statement, we consider the following parts one by one:
    \begin{proof}[Direction 1:]
    Given a decidable language $D$, the Language $C = \{ x | \exists y (\langle x,y \rangle \in D) \}$ formed is Turing recognizable. 
    \\
    \textbf{Proof:}
    Language D is Turing Decidable.
    
    
    \end{proof}
    
    \begin{proof}[Direction 2:]
    Given language $C$ such that $C = \{ x | \exists y (\langle x,y \rangle \in D) \}$ is Turing recognizable, prove that $D$ is Turing decidable.
    \\
    \textbf{Proof:}
    
    
    
    \end{proof}
    
    On combining both proven statements in Direction 1 and Direction 2, we can say that given language $C = \{ x | \exists y (\langle x,y \rangle \in D) \}$ is Turing recognizable if language $D$ is Turing Decidable.
    \begin{center}
        Hence Proved
    \end{center}
    
    We need to prove both directions. To handle the easier one first, assume that the decidable language D
exists. A TM recognizing C operates on input x by going through each possible string y and testing whether
hx, yi ∈ D. If such a y is ever found, accept; if not, just continue searching.
For the other direction, assume that C is recognized by TM, denoted by M. Define the language D to
be {hx, yi | M accepts x within |y| steps}. Language D is decidable since one can run M for y steps and
accept iff M has accepted. If x ∈ C, then M accepts x within some number of steps, so hx, yi ∈ D for some
sufficiently long y, but if x /∈ C then hx, yi ∈/ C for any y.
    %%%%%%%%%%%%%%%%%%%%%%%%%%%%%%%
        sol
    \end{proof}
\end{solution}
