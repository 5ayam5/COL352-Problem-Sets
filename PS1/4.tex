\begin{solution}{Question 4}\label{ques:4}
    \begin{question}
    Design an algorithm that takes as input the descriptions of two DFAs, $D_1$ and $D_2$, and
    determines whether they recognize the same language.
    \end{question}
    \tcblower{}
    \begin{proof}[Solution]
        Let $D_1$ and $D_2$ be two DFAs s.t 
        \[L(D_1) = L_1 \]
        \[L(D_2) = L_2 \]
        If languages $L_1$ and $L_2$ are same then following must hold true
        \begin{equation}
            L_1 \cap L_2^C = \{\phi\}
        \end{equation}
        and
        \begin{equation}
            L_1^C \cap L_2 = \{\phi\}
        \end{equation}
        As regular language is closed under intersection and complement we can find DFAs for languages in eq (2) and eq (3) using $D_1$ and $D_2$. Language accepted by a DFA is ${\phi}$ only when any final state is not reachable from the start state.
        
        %describe object DFA used in algo
        
        \begin{algorithm}[H]
            \caption{Finding DFA of complement of langauge accepted by given DFA $D$}
            \begin{algorithmic}[1]
                \Procedure{complement}{$D$} 
                    \State $DC \gets$ new DFA
                    \State $DC.Q \gets D.Q$
                    \State $DC.q_0 \gets D.q_0$
                    \State $DC.\delta \gets D.\delta$
                    \State $DC.final \gets D.Q \setminus$ $D.f$
                    \State{\Return DC}
                \EndProcedure{}
            \end{algorithmic}
        \end{algorithm}
        
        DFA of intersection of two languages $L_1$ and $L_2$ accepted by $D_1$ and $D_2$ respectively is given by product of DFAs $D_1$ and $D_2$. We find out the product DFA using the following algorithm.
        
        \begin{algorithm}[H]
            \caption{Finding product DFA of two given DFAs}
            \begin{algorithmic}[1]
                \Procedure{product}{$D_1, D_2$}
                    \State $D \gets$ new DFA
                    \State $D.Q \gets D_1.Q \times D_2.Q$
                    \State $D.q_0 \gets (D_1.q_0, D_2.q_0)$
                    \State $D.F \gets D_1.F \times D_2.F$
                    \ForAll{$(q_i, q_j) \in D.Q$}
                        \ForAll{$a \in D.\sum$}
                            \State $D.\delta((q_i, q_j), a) \gets (D_1.\delta(q_i, a), D_2.\delta(q_j, a))$
                        \EndFor{}
                    \EndFor{}
                    \State{\Return{$D$}}
                \EndProcedure{}
            \end{algorithmic}
        \end{algorithm}
        
        To check whether the language of DFA $D$ is $\{\phi\}$, we will do BFS on $D$ to check if the final state is reachable from start state in some transitions. If final state is not reachable then we can say that language of DFA is $\{\phi\}$
        
        \begin{algorithm}[H]
            \caption{Checking if any final state is reachable from start state in given DFA $D$}
            \begin{algorithmic}[1]
                \Procedure{isLanguageEmpty}{$D$}
                    \State visited $\gets$ init map with every state in $D.Q$ mapping to $0$
                    \State queue$\gets [D.q_0]$
                    \State visited[$D.q_0$] $\gets 1$
                    \State
                    \While{len(queue) $!= 0$}
                        \State state $\gets$ queue.pop(0)
                        \ForAll{$a$ in $D.\sum$}
                            \If{not visited$[D.\delta(state, a)]$}
                                \State queue.append($D.\delta(state, a)$)
                                \State visited$[D.\delta(state, a)]$ = 1
                            \EndIf{}
                        \EndFor{}
                    \EndWhile
                    \State
                    \State emptyLanguage $\gets 1$
                    \ForAll{$q$ in $D.F$}
                        \If{visited$[q]$}
                            \State emptyLanguage $\gets 0$
                            \State break
                        \EndIf{}
                    \EndFor{}
                    \State 
                    \State{\Return emptyLanguage}
                \EndProcedure{}
            \end{algorithmic}
        \end{algorithm}
        
        Using the above three procedures and stated algorithm we can check if two DFAs have same language. Following is the pseudo code for the algorithm.
        
        \begin{algorithm}[H]
            \caption{Checking if DFA $D_1$ and $D_2$ recognize same language}
            \begin{algorithmic}[1]
                \Procedure{sameLanguage}{$D_1, D_2$} 
                    \State $DFA_1 \gets \textsc{PRODUCT}(D_1, \textsc{COMPLEMENT}(D_2))$
                    \State $DFA_2 \gets \textsc{PRODUCT}(\textsc{COMPLEMENT}(D_1), D_2)$
                    \State 
                    \If{$\textsc{EMPTYLANGUAGE}(DFA_1)\ and\ \textsc{EMPTYLANGUAGE}(DFA_2)$}
                        \State{\Return true}
                    \EndIf{}
                    \State{\Return false}
                \EndProcedure{}
            \end{algorithmic}
        \end{algorithm}
        
    \end{proof}
\end{solution}
